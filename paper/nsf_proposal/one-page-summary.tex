\documentclass[11pt]{article}

%==============Packages & Commands==============
\usepackage{graphicx}
\usepackage{subfiles}
\usepackage{rotating}
\usepackage{color,soul}
\usepackage{tikz}
\usepackage{booktabs}
\usepackage{siunitx}
%%%<
\usepackage{threeparttable} %tablenote
\usepackage{verbatim}
\usepackage{geometry}     
\geometry{top=1in} 
\usepackage{graphicx}          
\usepackage{amssymb}
\usepackage[ruled,vlined]{algorithm2e}
\usetikzlibrary{arrows}
\usepackage{alltt}
\usepackage[T1]{fontenc}
\usepackage[utf8]{inputenc}
\usepackage{indentfirst}
\usepackage[natbib,authordate,backend=biber,sorting=nyt]{biblatex-chicago} %For apsr with natbib compatbility
\addbibresource{references.bib}
\usepackage{changepage}
\usepackage{setspace}
%\doublespacing
\usepackage{booktabs} % For tables
\usepackage{rotating} % For sideways tables/figures
\usepackage{amsmath}
\usepackage{multirow}
\usepackage{color}
\usepackage{dcolumn}
\usepackage{comment}
\usepackage{pgf}
\usepackage{xcolor, colortbl}
\usepackage{array}
\usepackage{subcaption}

\newtheorem{hyp}{Hypothesis}
\newtheorem{subhyp}{Hypothesis}
   \renewcommand\thesubhyp{\thehyp\alph{subhyp}}

\newcolumntype{d}[1]{D{.}{\cdot}{#1}}
\newcolumntype{.}{D{.}{.}{-1}}
\newcolumntype{3}{D{.}{.}{3}}
\newcolumntype{4}{D{.}{.}{4}}
\newcolumntype{5}{D{.}{.}{5}}
\usepackage{float}
% \usepackage[hyphens]{url}
\usepackage{fancyhdr}
\urlstyle{same}
\usepackage{times}

\usepackage{lscape}
\newcommand{\fnote}[1]{\footnote{\normalsize{\begin{doubledspace}#1\end{doubledspace}}}} % 12 pt, double spaced footnotes
\newenvironment{fignote}{\begin{quote}\footnotesize}{\end{quote}}

\newcommand\independent{\protect\mathpalette{\protect\independenT}{\perp}}
\def\independenT#1#2{\mathrel{\rlap{$#1#2$}\mkern2mu{#1#2}}}
\newcommand{\N}{\mathcal{N}}
\newcommand{\Y}{\bm{\mathcal{Y}}}
\newcommand{\bZ}{\bm{Z}}


\usepackage{tabularray}
\usepackage{float}
\usepackage{graphicx}
\usepackage{codehigh}
\usepackage[normalem]{ulem}
\UseTblrLibrary{booktabs}
\UseTblrLibrary{siunitx}
\newcommand{\tinytableTabularrayUnderline}[1]{\underline{#1}}
\newcommand{\tinytableTabularrayStrikeout}[1]{\sout{#1}}
\NewTableCommand{\tinytableDefineColor}[3]{\definecolor{#1}{#2}{#3}}

\usepackage[colorlinks = TRUE, urlcolor = black, linkcolor = black, citecolor = black, pdfstartview = FitV]{hyperref}

\title{How Does Public Opinion Affect Climate Change Policies?
Constructing Measures of Climate Change Public Concern and Polarization and Testing Their Effects on Climate Policy Outputs
}

\author{Yuehong Cassandra Tai\thanks{Postdoctoral Scholar, Center for Social Data Analytics, Pennsylvania State University, United States.}% \texttt{\href{mailto:yhcasstai@psu.edu}{yhcasstai@psu.edu}}. Phone: (319)512-6226. B001 Sparks, Pennsylvania State University, University Park, PA 16802}
\and Xun Cao \thanks{Professor, Department of Political Science, Pennsylvania State University, United States.}% , \texttt{\href{mailto:xuc11@psu.edu}{xuc11@psu.edu}}. Phone: (814) 865-8749. 310 Pond Lab, Pennsylvania State University, University Park, PA 16802}
\and Frederick Solt\thanks{Professor, Department of Political Science, University of Iowa, United States.}% , \texttt{\href{mailto:frederick-solt@uiowa.edu}{frederick-solt@uiowa.edu}}. Phone: 319-335-2340. 324 Schaeffer Hall (SH) Iowa City, IA 52242 United States}
}

\date{}
\begin{document}




\maketitle
\newpage
%This first paragraph demonstrates the importance of understanding how public opinion about climate change may affect politics and policy.
Effectively addressing climate change necessitates understanding the interplay between policy, politics, and public opinion. Understanding how public opinion on climate change affects policy, especially across partisan lines, urban-rural divides, and gender differences, is crucial, as polarization challenges the implementation and sustainability of climate reforms. However, existing theories often prioritize collective action problems \citep{ostrom1990governing,barrett2003environment,nordhaus2015climate,keohane2016cooperation} and distributive politics \citep{colgan2021asset,aklin2020prisoners}, overlooking the critical role of public attitudes, despite their influence on government actions in democracies \citep{dahl1971polyarchy,burstein2003impact} and in authoritarian countries \citep{alkon2018pollution}. We analyzed 70 highly-cited articles on public opinion toward climate change, identified from the top 100 Web of Science and Google Scholar results (duplicates removed), spanning 1998 to 2022. Many of these studies relied on case-study or small-n designs, with nearly half examining four or fewer countries and years, and over a third focusing on a single year in a single country. This highlights the significant absence of comprehensive, longitudinal, and cross-national studies in the field, making it imperative to fill this critical research gap.

%The second paragraph discusses the problems with available data: sparse and incomparable, as always, but on this topic particularly sparse (extra-fragmented)

The reason for this oversight is the lack of systematic cross-country public opinion data on climate change. Comparative public opinion data are sparse and fragmented, with many countries and years lacking data, and existing data often being incomparable due to differences in items and interpretations. These challenges are particularly severe for climate change opinion. Unlike established concepts such as democratic support, trust, and gender egalitarianism, climate-related questions have only been surveyed sporadically since the 1990s, mainly in developing countries. Actually, even for OECD countries, due to the absence of comparative data, \citet{schaffer2022policymakers} had to use media coverage of climate issues as a proxy for public concerns over 15 years. Our preliminary data collection from 93 different survey datasets from 1992 to 2022 highlights this sparsity, with only 3,830 country-year-item observations representing 37\% of the total possible data.

%The third paragraph describes how we will overcome these problems: latent variable estimation, and with more data, stronger priors in data-poor countries, combining years or employing stronger priors about change over time, treating items as similar (rather than simply as alike or different) by nesting them in hierarchical models should help in a couple of different ways, and using a variety of approaches to estimate polarization.

To address these challenges, we will rely on the Dynamic Comparative Public Opinion model (DCPO) \citep{solt2020modeling} for latent variable estimation on climate change. The DCPO model has been widely used in cross-national opinion estimation \citep{woo2023public,humacrointerest,woo2023measuring} among other latent variable models \citep[e.g.,][]{claassen2019estimating, caughey2019policy, mcgann2019parallel, kolczynska2024modeling}. We plan to expand the survey source data both temporally and geographically by including more country-specific surveys. We will also improve the multiple-imputation routine by incorporating new techniques to ensure DCPO's estimates are comparable. This includes setting stronger priors for change over time and in data-poor countries, and nesting similar items in hierarchical models to address item-comparability issues. Additionally, we will develop and test various methods for generating polarization estimates.


%In summary, the proposed work will build crucial research infrastructure for cross-national work on climate change as well as for work on cross-national public opinion.


Our project aims to build crucial research infrastructure for cross-national work on climate change, polarization, and broader public opinion. We will disseminate the resulting estimates to researchers, educators, students, and policymakers worldwide through conference presentations, a symposium, scholarly publications, and a web interface. This work will enhance our understanding of the relationship between public opinion and climate policies, ultimately contributing to more informed and effective climate action.

\printbibliography

\end{document}